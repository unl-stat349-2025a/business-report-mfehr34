% Options for packages loaded elsewhere
\PassOptionsToPackage{unicode}{hyperref}
\PassOptionsToPackage{hyphens}{url}
\PassOptionsToPackage{dvipsnames,svgnames,x11names}{xcolor}
%
\documentclass[
  letterpaper,
  DIV=11,
  numbers=noendperiod]{scrreprt}

\usepackage{amsmath,amssymb}
\usepackage{iftex}
\ifPDFTeX
  \usepackage[T1]{fontenc}
  \usepackage[utf8]{inputenc}
  \usepackage{textcomp} % provide euro and other symbols
\else % if luatex or xetex
  \usepackage{unicode-math}
  \defaultfontfeatures{Scale=MatchLowercase}
  \defaultfontfeatures[\rmfamily]{Ligatures=TeX,Scale=1}
\fi
\usepackage{lmodern}
\ifPDFTeX\else  
    % xetex/luatex font selection
\fi
% Use upquote if available, for straight quotes in verbatim environments
\IfFileExists{upquote.sty}{\usepackage{upquote}}{}
\IfFileExists{microtype.sty}{% use microtype if available
  \usepackage[]{microtype}
  \UseMicrotypeSet[protrusion]{basicmath} % disable protrusion for tt fonts
}{}
\makeatletter
\@ifundefined{KOMAClassName}{% if non-KOMA class
  \IfFileExists{parskip.sty}{%
    \usepackage{parskip}
  }{% else
    \setlength{\parindent}{0pt}
    \setlength{\parskip}{6pt plus 2pt minus 1pt}}
}{% if KOMA class
  \KOMAoptions{parskip=half}}
\makeatother
\usepackage{xcolor}
\setlength{\emergencystretch}{3em} % prevent overfull lines
\setcounter{secnumdepth}{5}
% Make \paragraph and \subparagraph free-standing
\ifx\paragraph\undefined\else
  \let\oldparagraph\paragraph
  \renewcommand{\paragraph}[1]{\oldparagraph{#1}\mbox{}}
\fi
\ifx\subparagraph\undefined\else
  \let\oldsubparagraph\subparagraph
  \renewcommand{\subparagraph}[1]{\oldsubparagraph{#1}\mbox{}}
\fi


\providecommand{\tightlist}{%
  \setlength{\itemsep}{0pt}\setlength{\parskip}{0pt}}\usepackage{longtable,booktabs,array}
\usepackage{calc} % for calculating minipage widths
% Correct order of tables after \paragraph or \subparagraph
\usepackage{etoolbox}
\makeatletter
\patchcmd\longtable{\par}{\if@noskipsec\mbox{}\fi\par}{}{}
\makeatother
% Allow footnotes in longtable head/foot
\IfFileExists{footnotehyper.sty}{\usepackage{footnotehyper}}{\usepackage{footnote}}
\makesavenoteenv{longtable}
\usepackage{graphicx}
\makeatletter
\def\maxwidth{\ifdim\Gin@nat@width>\linewidth\linewidth\else\Gin@nat@width\fi}
\def\maxheight{\ifdim\Gin@nat@height>\textheight\textheight\else\Gin@nat@height\fi}
\makeatother
% Scale images if necessary, so that they will not overflow the page
% margins by default, and it is still possible to overwrite the defaults
% using explicit options in \includegraphics[width, height, ...]{}
\setkeys{Gin}{width=\maxwidth,height=\maxheight,keepaspectratio}
% Set default figure placement to htbp
\makeatletter
\def\fps@figure{htbp}
\makeatother
\newlength{\cslhangindent}
\setlength{\cslhangindent}{1.5em}
\newlength{\csllabelwidth}
\setlength{\csllabelwidth}{3em}
\newlength{\cslentryspacingunit} % times entry-spacing
\setlength{\cslentryspacingunit}{\parskip}
\newenvironment{CSLReferences}[2] % #1 hanging-ident, #2 entry spacing
 {% don't indent paragraphs
  \setlength{\parindent}{0pt}
  % turn on hanging indent if param 1 is 1
  \ifodd #1
  \let\oldpar\par
  \def\par{\hangindent=\cslhangindent\oldpar}
  \fi
  % set entry spacing
  \setlength{\parskip}{#2\cslentryspacingunit}
 }%
 {}
\usepackage{calc}
\newcommand{\CSLBlock}[1]{#1\hfill\break}
\newcommand{\CSLLeftMargin}[1]{\parbox[t]{\csllabelwidth}{#1}}
\newcommand{\CSLRightInline}[1]{\parbox[t]{\linewidth - \csllabelwidth}{#1}\break}
\newcommand{\CSLIndent}[1]{\hspace{\cslhangindent}#1}

\KOMAoption{captions}{tableheading}
\makeatletter
\makeatother
\makeatletter
\@ifpackageloaded{bookmark}{}{\usepackage{bookmark}}
\makeatother
\makeatletter
\@ifpackageloaded{caption}{}{\usepackage{caption}}
\AtBeginDocument{%
\ifdefined\contentsname
  \renewcommand*\contentsname{Table of contents}
\else
  \newcommand\contentsname{Table of contents}
\fi
\ifdefined\listfigurename
  \renewcommand*\listfigurename{List of Figures}
\else
  \newcommand\listfigurename{List of Figures}
\fi
\ifdefined\listtablename
  \renewcommand*\listtablename{List of Tables}
\else
  \newcommand\listtablename{List of Tables}
\fi
\ifdefined\figurename
  \renewcommand*\figurename{Figure}
\else
  \newcommand\figurename{Figure}
\fi
\ifdefined\tablename
  \renewcommand*\tablename{Table}
\else
  \newcommand\tablename{Table}
\fi
}
\@ifpackageloaded{float}{}{\usepackage{float}}
\floatstyle{ruled}
\@ifundefined{c@chapter}{\newfloat{codelisting}{h}{lop}}{\newfloat{codelisting}{h}{lop}[chapter]}
\floatname{codelisting}{Listing}
\newcommand*\listoflistings{\listof{codelisting}{List of Listings}}
\makeatother
\makeatletter
\@ifpackageloaded{caption}{}{\usepackage{caption}}
\@ifpackageloaded{subcaption}{}{\usepackage{subcaption}}
\makeatother
\makeatletter
\@ifpackageloaded{tcolorbox}{}{\usepackage[skins,breakable]{tcolorbox}}
\makeatother
\makeatletter
\@ifundefined{shadecolor}{\definecolor{shadecolor}{rgb}{.97, .97, .97}}
\makeatother
\makeatletter
\makeatother
\makeatletter
\makeatother
\ifLuaTeX
  \usepackage{selnolig}  % disable illegal ligatures
\fi
\IfFileExists{bookmark.sty}{\usepackage{bookmark}}{\usepackage{hyperref}}
\IfFileExists{xurl.sty}{\usepackage{xurl}}{} % add URL line breaks if available
\urlstyle{same} % disable monospaced font for URLs
\hypersetup{
  pdftitle={Evaluation of },
  pdfauthor={Author Name},
  colorlinks=true,
  linkcolor={blue},
  filecolor={Maroon},
  citecolor={Blue},
  urlcolor={Blue},
  pdfcreator={LaTeX via pandoc}}

\title{Evaluation of}
\author{Author Name}
\date{}

\begin{document}
\maketitle
\ifdefined\Shaded\renewenvironment{Shaded}{\begin{tcolorbox}[breakable, borderline west={3pt}{0pt}{shadecolor}, interior hidden, boxrule=0pt, sharp corners, frame hidden, enhanced]}{\end{tcolorbox}}\fi

\renewcommand*\contentsname{Table of contents}
{
\hypersetup{linkcolor=}
\setcounter{tocdepth}{2}
\tableofcontents
}
\bookmarksetup{startatroot}

\hypertarget{preface}{%
\chapter*{Preface}\label{preface}}
\addcontentsline{toc}{chapter}{Preface}

\markboth{Preface}{Preface}

You are a data scientist for a mid-sized business, in a small group of
3-4 data scientists. You've been tasked with creating a report
evaluating a scenario for your business. Your colleagues will also be
evaluating the same scenario, and your reports will be used in aggregate
to determine a consensus (or lack thereof) on the company's action. The
reports will also be used to inform downsizing that is rumored to be
coming - you want to ensure your report is better than your peers so
that you aren't as easy to cut.

You may talk to your peers who are assigned the same scenario, but you
do not want to collaborate too closely, lest you both become targets of
the rumored layoffs.

\begin{center}\rule{0.5\linewidth}{0.5pt}\end{center}

I've scaffolded this report for you to make this process easier - as we
talk about different sections of a report in class and read about how to
create similar sections, you will practice by writing the equivalent
section of your report.

The basic steps for this task are as follows:

\begin{itemize}
\item
  Identify the research question from the business question
\item
  Identify data set(s) which are (1) publicly available (you don't have
  a budget to pay for private data) and (2) relevant to your task

  \begin{itemize}
  \tightlist
  \item
    (HW Week 6) Document your data sets in \texttt{draft-data-doc.qmd}
  \end{itemize}
\item
  Conduct a statistical analysis to support your answer to your research
  and business questions

  \begin{itemize}
  \item
    Write a methods section for your business report corresponding to
    your statistical analysis
  \item
    (HW Week 9) Draft of results section of business report with
    relevant graphics/visual aids in \texttt{draft-results.qmd}
  \end{itemize}
\item
  Write your report

  \begin{itemize}
  \item
    (HW Week 10) Draft of Intro/Conclusion sections in
    \texttt{draft-intro-conclusions.qmd}
  \item
    (HW Week 11) Draft of Executive summary section in
    \texttt{draft-exec-summary.qmd}
  \end{itemize}
\item
  Revise your report

  \begin{itemize}
  \item
    (HW Week 12 -- not turned in) Revise your report
  \item
    (HW Week 13) - Rough draft of report due. Create one or more qmd
    files for your report (you can overwrite or delete intro.qmd and
    summary.qmd), include the names of each file (in order) in
    \texttt{\_quarto.yml}. You should use references (edit
    references.bib and use pandoc citations). Make sure your report
    compiles and looks reasonable in both html and pdf.
  \item
    Develop a presentation to go along with your report (Week 13).
    Create slides for your report using quarto.
  \end{itemize}
\item
  Peer revise reports

  \begin{itemize}
  \item
    Peer revise reports
  \item
    (HW Week 14) - Make edits to your report from comments received from
    peer review
  \end{itemize}
\item
  Final report \& presentation due
\end{itemize}

\bookmarksetup{startatroot}

\hypertarget{introduction}{%
\chapter{Introduction}\label{introduction}}

This is a book created from markdown and executable code.

See Knuth (1984) for additional discussion of literate programming.

\bookmarksetup{startatroot}

\hypertarget{summary}{%
\chapter{Summary}\label{summary}}

In summary, this book has no content whatsoever.

\bookmarksetup{startatroot}

\hypertarget{references}{%
\chapter*{References}\label{references}}
\addcontentsline{toc}{chapter}{References}

\markboth{References}{References}

\hypertarget{refs}{}
\begin{CSLReferences}{1}{0}
\leavevmode\vadjust pre{\hypertarget{ref-USEnergy}{}}%
{``Homepage - u.s. Energy Information Administration (EIA).''} n.d.
\url{https://www.eia.gov/index.php}.

\leavevmode\vadjust pre{\hypertarget{ref-knuth84}{}}%
Knuth, Donald E. 1984. {``Literate Programming.''} \emph{Comput. J.} 27
(2): 97--111. \url{https://doi.org/10.1093/comjnl/27.2.97}.

\leavevmode\vadjust pre{\hypertarget{ref-usdepartmentofcommerceNationalDataBuoy}{}}%
US Department of Commerce, National Oceanic, and Atmospheric
Administration. n.d. {``National Data Buoy Center.''}
\url{https://www.ndbc.noaa.gov/}.

\end{CSLReferences}

\cleardoublepage
\phantomsection
\addcontentsline{toc}{part}{Appendices}
\appendix

\hypertarget{draft-data-documentation}{%
\chapter{Draft: Data Documentation}\label{draft-data-documentation}}

\hypertarget{water-temperature-data}{%
\section{Water Temperature Data}\label{water-temperature-data}}

The \textbf{Historical Dauphin Island Meteorological Data Set} will be
used to analyze water temperature data for this business report.

The data was collected by the National Data Buoy Center (NDBC), a branch
of the National Oceanic and Atmospheric Administration (NOAA). It was
collected to record hourly metrics for the Dauphin Island coast. The
NDBC used one of their many Moored Buoys to record data on barometric
pressure, wind direction/speed/gust, air/sea temperature, and wave
energy spectra. This specific buoy is located right off the east coast
of Dauphin Island, AL.

There are three data sets that have the exact same data and structure
that I will be combining to analyze the trends of water temperature.
These data sets are separated by year, from the years 2022, 2023, and
2024. Data was collected every six minutes, every day of the year
(starting in October for the 2022 data), and recorded as a new instance.

The files are in a tab delimited .txt format. Missing data points are
denoted by values of 99.00 or 999.0. The fields in this data set are:

\begin{itemize}
\tightlist
\item
  \textbf{YY} = year
\item
  \textbf{MM} = month
\item
  \textbf{DD} = day
\item
  \textbf{hh} = hour
\item
  \textbf{mm} = minute
\item
  \textbf{WDIR} = wind direction (degrees clockwise from North)
\item
  \textbf{WSPD} = wind speed (m/s)
\item
  \textbf{GST} = peak 5 or 8 second gust speed (m/s)
\item
  \textbf{WVHT} = significant wave height (m)
\item
  \textbf{DPD} = dominant wave period (sec)
\item
  \textbf{APD} = average wave period (sec)
\item
  \textbf{MWD} = direction from which waves at dominant period are
  coming (degrees clockwise from North)
\item
  \textbf{PRES} = sea level pressure (hPa, hectopascal)
\item
  \textbf{ATMP} = air temperature (degrees celsius)
\item
  \textbf{WTMP} = sea surface temperatrue (degrees celsius)
\item
  \textbf{DEWP} = dewpoint temperature (degrees celsius)
\item
  \textbf{VIS} = station visibility (nautical miles)
\item
  \textbf{TIDE} = water level above or below mean lower low water (ft)
\end{itemize}

The NDBC has implemented several automated processes to check quality
control. When a system is detected as defective, its data is immediately
taken down from the web site. Measurements from duplicate sensors are
checked to make sure they match.

The NOAA has released all their data publicly with no restrictions on
usage.

US Department of Commerce and Administration (n.d.)

\hypertarget{energy-data}{%
\section{Energy Data}\label{energy-data}}

The \textbf{Historical SOCO Energy Data Set} will be used to analyze
energy demand, generation, and interchange data for this report.

The data comes from the U.S. Energy Information Administration (EIA),
but was originally collected by the balancing authority Southern Company
(SOCO). SOCO is the largest energy company in the Southeast region of
the United States. They collected this data for the purposes of
balancing and forecasting electricity supply, demand, and interchange in
real time.

This data set houses \emph{daily} data on electricity in the Southeast.
The data was collected from 3/1/2024 to 2/27/2025.

The data file is set up as a .csv file. The EIA performed imputations on
extreme and missing values before making the data available, so there
are no missing values.

The fields included are:

\begin{itemize}
\tightlist
\item
  \textbf{BA Code}: balancing authority code (SOCO is the only value in
  this dataset)
\item
  \textbf{Timestamp}: date when data was collected (MM/DD/YYYY)
\item
  \textbf{Demand}: the amount of electricity load within SOCO's electric
  system (MWh)
\item
  \textbf{Demand Forecast}: electricity demand forecast for the next day
  (MWh)
\item
  \textbf{Net Generation}: the metered output of electric generating
  units in the electric system (MWh)
\item
  \textbf{Total Interchange}: The net metered tie line (a transmission
  line connecting power systems) flow from one BA to another
  interconnected BA (MWh)
\end{itemize}

The quality of the data is uncertain. The data is preliminary and is
made available ``as-is'' by the EIA. There were a few
processes/imputations done by the EIA to ensure minimum data quality,
but the EIA is not responsible for a reliance on the data.

Since the EIA is a government program, the data can be freely used while
properly citing.

{``Homepage - u.s. Energy Information Administration (EIA)''} (n.d.)

\hypertarget{draft-results}{%
\chapter{Draft: Results}\label{draft-results}}

\hypertarget{draft-introconclusions}{%
\chapter{Draft: Intro/Conclusions}\label{draft-introconclusions}}

\hypertarget{draft-executive-summary}{%
\chapter{Draft: Executive Summary}\label{draft-executive-summary}}



\end{document}
